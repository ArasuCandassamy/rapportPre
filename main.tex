\documentclass[a4paper,12pt,twoside]{report}
\usepackage[utf8]{inputenc}
\usepackage{hyperref}
\usepackage{graphicx}
\usepackage{comment}
\usepackage{geometry}
\usepackage{color}
\usepackage{fancyhdr}
\usepackage{tikz}
\usetikzlibrary{patterns}
\usepackage{titlesec}
\usepackage{lmodern}
\usepackage{setspace}
\usepackage{framed}
\usepackage{lipsum}
\usepackage{indentfirst}
\usepackage{amsmath,amssymb,enumerate,pgf}
\usepackage{physics}
\usepackage{amsthm}
\usepackage{caption}
\usepackage{subcaption}
\usepackage{tcolorbox}

\tcbuselibrary{skins, breakable}



\usepackage[
	backend=bibtex,
	style=alphabetic,
	sorting=ynt
]{biblatex}
\addbibresource{sample.bib}

\DefineBibliographyStrings{english}{%
  bibliography = {References},
}

\newcommand{\mtn}{\mathbb{N}}
\newcommand{\mtns}{\mathbb{N}^*}
\newcommand{\mtz}{\mathbb{Z}}
\newcommand{\mtr}{\mathbb{R}}
\newcommand{\mtk}{\mathbb{K}}
\newcommand{\mtq}{\mathbb{Q}}
\newcommand{\mtc}{\mathbb{C}}
\newcommand{\mch}{\mathcal{H}}
\newcommand{\mcp}{\mathcal{P}}
\newcommand{\mcb}{\mathcal{B}}
\newcommand{\mcl}{\mathcal{L}}
\newcommand{\mcm}{\mathcal{M}}
\newcommand{\mcc}{\mathcal{C}}
\newcommand{\mcmn}{\mathcal{M}}
\newcommand{\mcmnr}{\mathcal{M}_n(\mtr)}
\newcommand{\mcmnk}{\mathcal{M}_n(\mtk)}
\newcommand{\mcsn}{\mathcal{S}_n}
\newcommand{\mcs}{\mathcal{S}}
\newcommand{\mcd}{\mathcal{D}}
\newcommand{\dif}{\mathrm{d}}
\newcommand{\mcsns}{\mathcal{S}_n^{++}}
\newcommand{\glnk}{GL_n(\mtk)}
\newcommand{\mnr}{\mathcal{M}_n(\mtr)}
\DeclareMathOperator{\ch}{ch}
\DeclareMathOperator{\sh}{sh}
\DeclareMathOperator{\vect}{vect}
\DeclareMathOperator{\card}{card}
\DeclareMathOperator{\comat}{comat}
\DeclareMathOperator{\imv}{Im}
\DeclareMathOperator{\rang}{rg}
\DeclareMathOperator{\Fr}{Fr}
\DeclareMathOperator{\diam}{diam}
\DeclareMathOperator{\supp}{supp}
\DeclareMathOperator{\mat}{mat}
\newcommand{\veps}{\varepsilon}
\newcommand{\mcu}{\mathcal{U}}
\newcommand{\mcun}{\mcu_n}
\newcommand{\dis}{\displaystyle}
\newcommand{\croouv}{[\![}
\newcommand{\crofer}{]\!]}
\newcommand{\rab}{\mathcal{R}(a,b)}
\newcommand{\pss}[2]{\langle #1,#2\rangle}
\DeclareMathOperator{\sign}{sign}

\newtheorem{theorem}{Theorem}[section]
\newtheorem{corollary}{Corollary}[theorem]
\newtheorem{lemma}[theorem]{Lemma}


\newcommand{\ReportTitle}{Contribution to the development of Psydac
}
\author{Arasu Candassamy}

\geometry{top=2cm, bottom=2cm, left=1.5cm, right=1.5cm, headheight=15pt}

\pagestyle{fancy}
\fancyhf{} 
\fancyhead[CE]{\rightmark}   
\fancyhead[CO]{\ReportTitle} 
\fancyfoot[C]{\textbf{Arasu Candassamy} / Max-Planck-Institute Für Plasmaphysik \\
\textcolor{red}{\textbf{Non-confidential report and Can be published on the Internet}}}
\fancyfoot[R]{\thepage}
\renewcommand{\footrulewidth}{0.5pt}

\fancypagestyle{plain}{
    \fancyhf{} 
    \fancyhead[CE]{\rightmark}   
    \fancyhead[CO]{\ReportTitle} 
    \fancyfoot[C]{\textbf{Arasu Candassamy} / Max-Planck-Institute Für Plasmaphysik \\
    \textcolor{red}{\textbf{Non-confidential report and Can be published on the Internet}}}
    \fancyfoot[R]{\thepage}
    \renewcommand{\footrulewidth}{0.5pt}
}

\fancypagestyle{test}{
    \fancyhf{} 
    \fancyhead[CE]{}   
    \fancyhead[CO]{} 
    \fancyfoot[C]{\textbf{Arasu Candassamy} / Max-Planck-Institute Für Plasmaphysik \\
    \textcolor{red}{\textbf{Non-confidential report and Can be published on the Internet}}}
    \fancyfoot[R]{\thepage}
    \renewcommand{\footrulewidth}{0.5pt}
}

% Permet à \leftmark et \rightmark de fonctionner correctement
\renewcommand{\chaptermark}[1]{\markright{#1}}
\renewcommand{\sectionmark}[1]{\markright{#1}{}}


\begin{document}
\thispagestyle{empty}

% Logos
\begin{minipage}{0.6\textwidth}
    \includegraphics[width=5cm]{logoENSTA.jpg} 
\end{minipage}
\hfill
\begin{minipage}{0.6\textwidth}
    \includegraphics[width=5cm]{logoMPI.png}
\end{minipage}

\vspace{2cm}

\begin{center}
    {\LARGE \textbf{Research Internship (PRe)}}
    
    \vspace{0.5cm}
    \textbf{Major: Applied Mathematics} \\
    \textbf{School year: 2024-2025}
    
    \vspace{2cm}
    
    {\Huge \textcolor{gray}{Contribution to the development of Psydac, a finite element library.}} \\
	{\Large \textcolor{gray}{Implementation of solid mechanics equations using Psydac and simulation on a supercomputer.}} \\
    
    \vspace{1.5cm}
        
    \begin{framed}
        \centering
        \textcolor{red}{\textbf{Non-confidential report and Can be published on the Internet}} 
    \end{framed}
\end{center}


\begin{minipage}[t]{0.5\textwidth}
	\begin{flushleft} 
    \textbf{Author} : Arasu Candassamy \\
    \vspace{1cm}
	\textbf{ENSTA Supervisor:}\\
	\textsc{Sonia Fliss}\\
	\href{mailto:sonia.fliss@ensta.fr}{sonia.fliss@ensta.fr}
	\end{flushleft}
\end{minipage}
~
\begin{minipage}[t]{0.4\textwidth}
	\begin{flushright} 
    \textbf{Year Group}: 2026 \\
    \vspace{1cm}
	\textbf{Host Organization Supervisor:}\\
	\textsc{Martin Campos-Pinto}\\
	\href{mailto:martin.campos-pinto@ipp.mpg.de}{martin.campos-pinto@ipp.mpg.de}\\
	\end{flushright}
\end{minipage} \\

\vspace{1cm}

\begin{center}
    \textbf{Internship} from 26/05/2025 to 22/08/2025\\
\textbf{Name of the host organization:} Max-Planck-Institute Für Plasmaphysik\\
\textbf{Address:} Boltzmannstrasse 2, 85748 Garching bei München
\end{center}


\newpage\null

\newpage
\section*{Abstract}
\addcontentsline{toc}{chapter}{Abstract}

\lipsum[1-2]

\vspace{1cm}

\underline{Keywords}:

\vspace{1cm}

\section*{Résumé}

\lipsum[1-2]

\vspace{1cm}

\underline{Mots-clés}:

\newpage\null

\newpage

\section*{Acknowledgement}
\addcontentsline{toc}{chapter}{Acknowledgement}
\lipsum[1-2]


\newpage
\tableofcontents
\thispagestyle{test}


\newpage
\chapter{Introduction}

The Max-Planck Institute for Plasma Physics (IPP), located in Garching and Greifswald, is a renowned research institute in Plasma physics and Nuclear fusion. Founded under the Max-Planck Society in 1960, this institute has a long tradition of scientific excellence and innovation. The Department of Numerical Methods for Plasma Physics (NMPP), where my internship took place, plays a crucial role in developing advanced simulation techniques to understand and predict the behavior of plasmas in complex environments (in tokamaks or stelerators). \\

The Finite Element Method (FEM) is a widely used numerical technique for solving Partial Differential Equations (PDEs). It is particularly well-suited for addressing plasma physics problems due to its ability to handle complex geometries and varied boundary conditions. FEM allows for the discretization of a continuous domain into a set of finite elements, thereby facilitating the numerical resolution of equations governing the physical phenomena under study. \\

In the context of electromagnetism, the Finite Element Exterior Calculus (FEEC) formulation is a powerful approach for solving Maxwell's equations. This method is particularly suitable for problems involving electromagnetic fields, as it preserves certain topological and physical properties of the original equations. FEEC provides a powerfull framework for plasma simulation, where electromagnetic interactions play a central role. \\

Psydac is an integrated development environment (IDE) specifically designed for simulation and data analysis in plasma physics. It offers a range of tools to facilitate the implementation of numerical methods, including FEM and FEEC frameworks. Psydac enables researchers to develop complex models, simulate them, and analyze results efficiently and intuitively. Psydac also stands out as an isogeometric analysis library, an approach that integrates geometric modeling and numerical analysis within a unified framework. Isogeometric analysis allows simulations to be performed directly on the geometric models used for design, eliminating the need to convert CAD models into traditional meshes for simulations. This provides increased accuracy and better integration between design and analysis, reducing errors and improving the efficiency of the simulation process. Using Psydac, researchers can leverage these capabilities to develop more accurate models and perform more reliable analyses, which is particularly advantageous in the complex field of plasma physics. \\

This internship report describes my experience and contributions within the NMPP Department at the IPP. Psydac should be able to solve any PDEs but it hasn't been tested on equations from solid mechanics. Then, during this internship, I have explored the linear elasticity problem, its simulation using Psydac and its running on Supercomputer Raven.

\chapter{Linear Elasticity}

\section{Physical Model and Strong Formulation}
The following presentation of the context of work and physical background comes from \cite{gould_introduction_2013} and \cite{ern_theory_2004}.
Let's denote by $\Omega \subset \mtr^3$ a deformable medium characterized by $\lambda$ and $\mu$ Lamé's coefficient ($\lambda, \mu > 0$). $\Omega$ is a bounded open domain with lipschitzian boundary. $\lambda$ is called Lamé's first coefficient, $\mu$ is the shear modulus. These coefficient are related to the Young modulus and Poisson coefficient of the medium. The deformable medium is under an external load $\vec f : \Omega \rightarrow \mtr^3$. Then, we are looking for $\vec u : \Omega \rightarrow \mtr^3$ the displacement field, when the equilibrium is reached. 

The material verifies the following equilibrium condition : 
\begin{equation}
\label{eq:equilibrium}
    \div \boldsymbol{\sigma} (\vec u) + \vec f = \vec 0 \text{ in } \Omega,
\end{equation}
where $\boldsymbol{\sigma} (\vec u)$ is the Cauchy Stress Tensor, defined by $\boldsymbol{\sigma} (\vec u) = C : \boldsymbol{\varepsilon} (\vec u)$. $C$ is the fourth-order stiffness tensor and $\boldsymbol{\varepsilon} (\vec u) := \frac{1}{2}(\grad \vec u + (\grad \vec u)^T)$ is the strain tensor. In the context of linear elascity, the relation between $\boldsymbol{\varepsilon} (\vec u)$ and $\boldsymbol{\sigma} (\vec u)$ is simpler : 

\begin{equation}
\label{eq:hooklaw}
    \boldsymbol{\sigma} (\vec u) = \lambda \Tr(\boldsymbol{\varepsilon}) I_3 + 2\mu \boldsymbol \varepsilon = \lambda (\div \vec u) I_3 + \mu (\grad \vec u + (\grad \vec u)^T) \text{ in } \Omega.
\end{equation}

For boundary conditions, there are two types of boundary conditions : Dirichlet conditions on the displacement field $\vec u$ and Traction conditions on the Cauchy Stress Tensor $\boldsymbol \sigma (\vec u)$. 


\begin{figure}[!h]
	\centering
	\includegraphics[width=0.61\linewidth]{omega_domain}
	\caption{Representation of a domain $\Omega$}
	\label{fig:omegadomain}
\end{figure}

Let's decompose $\partial \Omega$ as $\partial \Omega_D$ and $\partial \Omega_N$ such that $\partial \Omega_D \cup \partial \Omega_N = \partial \Omega$, $\partial \Omega_D \cap \partial \Omega_N = \emptyset$. Let's denote by $\vec t_N : \partial \Omega_N \rightarrow \mtr^3$ the normal traction force and $\vec n$ the normal vector of $\partial \Omega$ outside-oriented. We can consider the boundary displacement on $\partial \Omega_D$ as equal to $\vec 0$. 

In case of non-homogenous boundary conditions $\vec g : \partial \Omega_D \rightarrow \mtr^3$, we can solve the problem for $\vec u - \vec g$ by modifying the source term. Then, the solution of linear elasticity problem should verify :
$
\left\{ \begin{array}{ll}
\vec u = \vec 0 & \partial \Omega_D \\
\boldsymbol{\sigma} (\vec u) \cdot \vec n = \vec t_N & \partial \Omega_N
\end{array} \right.
$

Physically, $\vec f$ is a "volume force" on $\Omega$ and $\vec t_N$ is a "surface force". 
The strong formulation of this problem is : \\
For a given $\vec f \in \boldsymbol L^2(\Omega)$, $\vec t_N \in \boldsymbol L^2(\partial \Omega_N)$, find $\vec u \in \boldsymbol H^1(\Omega)$ such that : 

\begin{tcolorbox}
\begin{equation}
\label{eq:strong_form}
\left\{
\begin{aligned}
    & - \div \boldsymbol{\sigma} (\vec u) = \vec f & \text{ in } & \Omega \hspace{1cm} \text{(Equilibrium)}\\
    & \boldsymbol{\sigma} (\vec u) = \lambda (\div \vec u) I_3 + \mu (\grad \vec u + (\grad \vec u)^T) & \text{ in }& \Omega \hspace{1cm} \text{(Hook's law)} \\
    & \vec u = \vec 0 & \text{ in } &\partial \Omega_D \\
    & \boldsymbol{\sigma} (\vec u) \cdot \vec n = \vec t_N & \text{ in }& \partial \Omega_N
\end{aligned}
\right.
\end{equation}
\end{tcolorbox}

where $\boldsymbol L^2(\Omega) = \left(L^2(\Omega) \right)^3, \boldsymbol L^2(\partial \Omega_N) = \left(L^2(\partial \Omega_N) \right)^3, \boldsymbol H^1(\Omega) = \left(H^1(\Omega) \right)^3, $

\section{Pure displacement weak-formulation} \label{sec:Pure displacement weak-formulation}

To derive the weak formulation from \eqref{eq:strong_form}, let's introduce the following functionnal space : $\boldsymbol H^1_{\vec 0,D}(\Omega) = \left\{ \vec v \in \boldsymbol H^1(\Omega) \Big| \vec v_{|\partial\Omega_D} = \vec 0 \right\},$
which is a sub-hilbert space of $\boldsymbol H^1(\Omega)$ equipped with the same dot product (as a closed space of a Hilbert space). By taking the scalar product of the equilibrium equation from \eqref{eq:strong_form} with $\vec v \in \left(H^1_{\vec 0,D}(\Omega) \right)^3$ and applying a Green's formula, one gets that :
$\int_\Omega \boldsymbol{\sigma} (\vec u) : (\grad \vec v) \dif V - \int_{\partial\Omega} \left(\boldsymbol{\sigma} (\vec u).\vec n \right).\vec v \dif S = \int_\Omega \vec f . \vec v \dif V$

One can remark that with \eqref{eq:hooklaw}, the tensor $\boldsymbol \sigma (\vec u)$ is symetric. It implies that : 
$\boldsymbol \sigma (\vec u) : \grad \vec v = \boldsymbol \sigma(\vec u) : \boldsymbol \varepsilon (\vec v)$
Moreover, $\vec v \in \left(H^1_{\vec 0,D}(\Omega) \right)^3$, then $\vec v_{|\partial\Omega_D} = \vec 0$. So the weak formulation can be simplified into : 

\begin{equation}
\label{eq:weakdisplacementform}
\boxed{
\left\{
    \begin{aligned}
    &\text{Find } \vec u \in \boldsymbol H^1_{\vec 0,D}(\Omega) \text{ such that :}\\
    & a(\vec u,\vec v) = l(\vec v) \hspace{0.5cm} \forall \vec v \in \boldsymbol H^1_{\vec 0,D}(\Omega)
    \end{aligned}
\right.
}
\end{equation}

With : 
\begin{equation*}
    \begin{aligned}
        & a : \left\{
        \begin{aligned}
            &\boldsymbol H^1_{\vec 0,D}(\Omega) \times \boldsymbol H^1_{\vec 0,D}(\Omega)  \rightarrow \mtr \\
            &(\vec u,\vec v)  \longmapsto \int_\Omega \boldsymbol \sigma(\vec u) : \boldsymbol \varepsilon (\vec v) \dif V = \int_\Omega \lambda (\div \vec u) (\div \vec v) \dif V+ \int_\Omega 2\mu \boldsymbol \varepsilon(\vec u): \boldsymbol \varepsilon (\vec v)\dif V
        \end{aligned}
        \right. \\
    	& l : \left\{
        \begin{aligned}
            &\boldsymbol H^1_{\vec 0,D}(\Omega) \rightarrow \mtr \\
            &\vec v \longmapsto \int_\Omega \vec f . \vec v \dif V + \int_{\partial\Omega_N} \vec t_N.\vec v \dif S
        \end{aligned}
        \right.
    \end{aligned}
\end{equation*}
In continuum mechanics, \eqref{eq:weakdisplacementform} is called the Principle of Virtual Work. Let's look at the well-posedness of \eqref{eq:weakdisplacementform}. Before starting, let's remind some notations and lemmas. \\

$\norm{.}_{\boldsymbol H^1(\Omega)}$ is defined as : $\forall \vec v \in \boldsymbol H^1(\Omega), \norm{\vec v}_{\boldsymbol H^1(\Omega)} = \left(\sum_{i=1}^3 \norm{v}_{H^1(\Omega)} \right)^{1/2}$. 
\begin{lemma}
\begin{enumerate}
    \item $\displaystyle \forall \vec u \in \boldsymbol H^1(\Omega)$, $\int_\Omega (\div \vec u)^2 \dif V \leq 3 \norm{\vec u}_{\boldsymbol H^1(\Omega)}^2$
    \item $\displaystyle \forall \vec u \in \boldsymbol H^1(\Omega)$,$\boldsymbol \varepsilon(\vec u) : \boldsymbol \varepsilon(\vec u) \leq \grad \vec u : \grad \vec u$
    Then : $\int_\Omega \boldsymbol \varepsilon(\vec u) : \boldsymbol \varepsilon(\vec u) \dif\Omega \leq \int_\Omega \grad \vec u : \grad \vec u \dif \Omega \leq \norm{\vec u}_{\boldsymbol H^1(\Omega)}^2$
\end{enumerate}
\end{lemma}
\begin{proof}
    By calculus, or a detailed version in \cite{cinatl_finite}
\end{proof}

\begin{lemma}
\label{Korn1}
    (First Korn's inequality) \\
    Let $\Omega \subset \mtr^3$ a domain and denotes by $\norm{\boldsymbol \varepsilon(\vec u)}_{\boldsymbol L^2(\Omega)} := \left(\int_\Omega \boldsymbol \varepsilon(\vec u):\boldsymbol \varepsilon(\vec u)\right)^{1/2}$. Then, there exists $C > 0$ such that: $\forall \vec v \in \boldsymbol H^1_{\vec 0}(\Omega), \hspace{1cm} C\norm{\vec v}_{\boldsymbol H^1(\Omega)} \leq \norm{\boldsymbol \varepsilon(\vec v)}_{\boldsymbol L^2(\Omega)}$
\end{lemma}
\begin{proof}
    A proof can be found in \cite{brenner_mathematical_2008}
\end{proof}

\begin{lemma}
\label{Korn2}
    (Second Korn's inequality) \\
    Let $\Omega \subset \mtr^3$ a domain. Then, there exists $C > 0$ such that: 
    $\forall \vec v \in \boldsymbol H^1(\Omega), \hspace{1cm} C\norm{\vec v}_{\boldsymbol H^1(\Omega)} \leq \norm{\boldsymbol \varepsilon(\vec v)}_{\boldsymbol L^2(\Omega)} + \norm{\vec v}_{\boldsymbol L^2(\Omega)}$
    By Rellich theorem, $\boldsymbol H^1(\Omega)$ is compactly embedded in $\boldsymbol L^2(\Omega)$ as $\Omega$ is a open bounded domain . Then, one can show that there exists $C > 0$ such that:
    $\forall \vec v \in \boldsymbol H^1(\Omega), \hspace{1cm} C\norm{\vec v}_{\boldsymbol H^1(\Omega)} \leq \norm{\boldsymbol \varepsilon(\vec v)}_{\boldsymbol L^2(\Omega)}$
\end{lemma}
\begin{proof}
    A proof can be found in \cite{brenner_mathematical_2008}
\end{proof}


\noindent \textbf{Now, we can prove the well-posedness of \eqref{eq:weakdisplacementform} when $\left| \Omega_D \right| > 0$.}
\begin{itemize}
    \item $\left(\boldsymbol H^1_{\vec 0,D}(\Omega), \norm{.}_{\boldsymbol H^1(\Omega)} \right)$ is a Hilbert space.
    \item $a(.,.)$ is a bilinear form on $\boldsymbol H^1_{\vec 0,D}(\Omega) \times \boldsymbol H^1_{\vec 0,D}(\Omega)$ and $l(.)$ a linear form on $\boldsymbol H^1_{\vec 0,D}(\Omega)$.
    \item $l(.)$ is continuous : \begin{equation*}
    \begin{aligned}    
    \forall \vec v \in \boldsymbol H^1_{\vec 0,D}(\Omega), \left|l(\vec v)\right| 
    & \leq \left| \left(\vec f, \vec v \right)_{\boldsymbol L^2(\Omega)} + \left(\vec t_N, \vec v \right)_{\boldsymbol L^2(\partial \Omega_N)} \right| \\
    \left|l(\vec v)\right| 
    & \leq \left( \norm{\vec f}_{\boldsymbol L^2(\Omega)} + \norm{\gamma_0} \norm{\vec t_N}_{\boldsymbol L^2(\partial \Omega_N)}\right)\norm{\vec v}_{\boldsymbol H^1(\Omega)}\\
    \end{aligned}
    \end{equation*}
    \item $a(.,.)$ is continuous: 
    \begin{equation*}
        \begin{aligned}
        \forall \vec u, \vec v \in \boldsymbol H^1_{\vec 0,D}(\Omega), \left| a(\vec u, \vec v) \right| 
        & \leq \left| a(\vec u, \vec u) \right|^{1/2} \left| a(\vec v, \vec v) \right|^{1/2} & \text{By Cauchy-Schwarz Inequlity as $a(.,.)$ is} \\
        & & \text{a symetric and positive bilinear form} \\
        \left| a(\vec u, \vec v) \right| & \leq (3\lambda + 2\mu)\norm{\vec u}_{\boldsymbol H^1(\Omega)}\norm{\vec v}_{\boldsymbol H^1(\Omega)}.
        \end{aligned}
    \end{equation*}
    \item $a(.,.)$ is coercive: 
        \begin{itemize}
            \item If $\Omega_D = \Omega$, then $\boldsymbol H^1_{\vec 0,D}(\Omega) = \boldsymbol H^1_{\vec 0}(\Omega)$ and we can apply \textbf{Lemma \ref{Korn1}} : there exists $C > 0$ such that:
            \begin{equation*}
                \begin{aligned}
                    \forall \vec u \in \boldsymbol H^1_{\vec 0,D}(\Omega), \hspace{0.3cm} a(u,u) 
                    &= \int_\Omega \lambda (\div \vec u)^2 \dif V + \int_\Omega 2\mu \boldsymbol \varepsilon(\vec u): \boldsymbol \varepsilon (\vec u)\dif V \\
                    & \geq 2\mu \norm{\boldsymbol \varepsilon(\vec u)}_{\boldsymbol{L}^2(\Omega)} \\
                    a(u,u) & \geq 2\mu C \norm{\vec u}_{\boldsymbol{H}^1(\Omega)}^2
                \end{aligned}
            \end{equation*}
            \item If $\Omega_D \subsetneq \Omega$, we can now apply \textbf{Lemma \ref{Korn2}} : there exists $C >0$ such that:
            $\forall \vec u \in \boldsymbol H^1_{\vec 0,D}(\Omega), \hspace{0.3cm}a(u,u) \geq 2\mu C \norm{\vec u}_{\boldsymbol{H}^1(\Omega)}$
        \end{itemize}
\end{itemize}

Then, by Lax-Milgram theorem, the problem \eqref{eq:weakdisplacementform} is well-posed when $\left| \Omega_D \right| > 0$. \\
When, $\left| \Omega_D \right| = 0$, \eqref{eq:weakdisplacementform} turns into the\textbf{ Pure Traction Problem} : 

\begin{equation}
\label{eq:puretractionproblem}
\boxed{
\left\{
    \begin{aligned}
    &\text{Find } \vec u \in \boldsymbol H^1(\Omega) \text{ such that :}\\
    & a(\vec u,\vec v) = \int_\Omega \vec f . \vec v \dif V + \int_{\partial\Omega} \vec t_N.\vec v \dif S =:l(\vec v) \hspace{0.5cm} \forall \vec v \in \boldsymbol H^1(\Omega)
    \end{aligned}
\right.
}
\end{equation}

This problem isn't really well-posed. Regarding to the Rigid Displacement Field (translations and rotations) $\boldsymbol{\mathcal{R}} = \left\{\vec v \in \boldsymbol H^1(\Omega) \left| \exists x_1, x_2 \in \mtr^3, \vec v(x) = x_1 + x_2 \times x \right. \right\}$, $\forall \vec v \in \boldsymbol{\mathcal{R}}, \hspace{0.3cm} \div v = 0 \text{ and } \boldsymbol \varepsilon(\vec v) = 0.$
Then : $\forall \vec u \in \boldsymbol H^1(\Omega), \forall \vec v \in \boldsymbol{\mathcal{R}},\hspace{0.3cm} a(\vec u, \vec v)= 0.$
So, the weak formulation \eqref{eq:puretractionproblem} is solvable iff the following compatibility equality holds : $\forall \vec v \in \boldsymbol{\mathcal{R}}, \hspace{0.3cm}\int_\Omega \vec f . \vec v \dif V + \int_{\partial\Omega} \vec t_N.\vec v \dif S = 0,$
and in this case, there exists an unique solution $\vec u \in \hat{\boldsymbol H}^1(\Omega) := \left\{\vec v \in \boldsymbol H^1(\Omega) \left| \displaystyle \int_\Omega \vec v \dif V = \int_\Omega  \curl \vec v \dif V = \vec 0 \right. \right\}$ 
In case of \textbf{Pure Traction Problem}, we are looking for a solution modulo a rigid displacement field.
\textit{A more rigorous proof of the equivalence can be found in \cite{brenner_mathematical_2008} and \cite{chen_variational}.}


\section{Numerical simulation using \texttt{PSYDAC} on the Supercomputer Raven}

\texttt{PSYDAC} is high-level programming IDE and very intuitive to use. It is using \texttt{SymPDE} and \texttt{Pyccel}. \texttt{SymPDE} implements in \texttt{Python} abstract objects (Functionnal Space, diffential operators, linear and bilinear forms...). \texttt{Pyccel} is a transpiler, used to convert a \texttt{Python} Code into a \texttt{C} or \texttt{Fortran} code, in order to take advantage from code acceleration and parallelization. The figure \ref{fig:Dependancy_Psydac} summarize the dependancy between \texttt{PSYDAC}, \texttt{SymPDE} and \texttt{Pyccel}. A more detailled presentation of \texttt{PSYDAC} can be found in \cite{guclu_psydac_2022}.
\newpage 

\begin{figure}[!h]
    \centering
    \includegraphics[width=0.75\linewidth]{psydac_fonc_1.png}
    \caption{Dependancy graph of \texttt{PSYDAC}, from \cite{guclu_psydac_2022}}
    \label{fig:Dependancy_Psydac}
\end{figure}

Raven is a supercomputer located at the Max-Planck Institute for Plasma Physics. It is part of the High Performance Computing (HPC) infrastructure and is used for large-scale simulations and data analysis in plasma physics and related fields. I had the opportunity to run my simulations on Raven, which provided me with access to powerful computational resources and parallel processing capabilities.

\subsection{Dirichlet Homogeneous Boundary conditions on $\partial\Omega_D = \partial\Omega$} \label{subsec:Dirichlet Homogeneous Boundary conditions on}

In a first time, to get the grips with \texttt{PSYDAC} and the HPC, I have considered the case $\Omega = [0,1]^3$, $\partial \Omega_D = \partial\Omega$ and $\partial \Omega_N = \emptyset$. To verify if the numerical resolution of \texttt{PSYDAC} of \eqref{eq:weakdisplacementform} is correct, I have used the Method of Manufactured Solutions. The concept is quite simpler, we assume that a function $\vec u_e \in \boldsymbol H^1_{\vec 0}(\Omega)$ verifies the strong problem \eqref{eq:strong_form}, then we compute the source term $\vec f$ and we solve the \eqref{eq:weakdisplacementform} with $\vec f$ as a source term and we look at the error between the numerical solution and the theoretical one.
I have considered $\vec u_e$ defined as : $\vec u_e = \begin{pmatrix}
    0 \\
    0 \\
\sin{\left(\pi x \right)} \sin{\left(\pi y \right)} \sin{\left(\pi z \right)}
\end{pmatrix} \in \boldsymbol H^1_{\vec 0}(\Omega).$ Then, we can compute manually the source term $\vec f$ : $\vec f = \begin{pmatrix}
     - \pi^{2} \left(\lambda + \mu\right) \cos{\left(\pi x \right)} \sin{\left(\pi y \right)}  \cos{\left(\pi z \right)} \\
     - \pi^{2} \left(\lambda + \mu\right) \sin{\left(\pi x \right)} \cos{\left(\pi y \right)} \cos{\left(\pi z \right)} \\
      \pi^{2} \left(\lambda + 4 \mu\right) \sin{\left(\pi x \right)} \sin{\left(\pi y \right)} \sin{\left(\pi z \right)}
\end{pmatrix} \in \boldsymbol L^2(\Omega).$

To make the numerical simulation with \texttt{PSYDAC}, we have to choose two important parameters : $d$ the maximum degree of B-Splines functions that will approximate the solution and $n_{\text{cells}}$ is the number of cells in each dimension of $\Omega$ for the finite dimensional space. $\displaystyle h \propto \frac{1}{n_{\text{cells}}}$, an usual discretization parameters for finite dimensional spaces and $d$ will conduct the convergence speed of the numerical solution. 
By solving the problem Dirichlet Homogeneous Boundary conditions on the whole border, one can observe the following plots of the simulated $\vec u_{e,3}^h$ for different plane $z$. It is possible to plot also the error $\vec u_{e,3} - \vec u_{e,3}^h$.

\newpage

\begin{figure}
    \centering
    \includegraphics[width=0.8\linewidth]{3d_plots_degree_2_non_mixed_dirichlet_homogeneous_ncell=64.png}
    \caption{Plots comparison between the simulated and real $u_{e,3}$ with the error, for different plane $z$, with $d=2$ and $n_{\text{cell}} = 64$}
    \label{fig:3D_plots}
\end{figure}

We can observe that the numerical solution is "quite" near the real solution. To get a better estimation of convergence rate, let's look at the Céa's lemma in this case.

Denotes by $\boldsymbol V^h \subset \boldsymbol H^1(\Omega)$ the discrete finite dimensional space, $\vec u \in \boldsymbol H^{s+1}(\Omega) \subset \boldsymbol H^1(\Omega), 1 \leq s \leq d+1$ the real solution solution and $\vec u^h \in \boldsymbol V^h$ the discrete solution.
Then, the Céa's lemma gives the following estimates : 
$\norm{\vec u - \vec u^h}_{\boldsymbol H^1(\Omega)} \leq \frac{3\lambda + 2\mu}{2\mu C} \inf_{\vec v^h \in \boldsymbol H^1(\Omega)} \norm{\vec u - \vec v^h}_{\boldsymbol H^1(\Omega)}$
\textit{where $C > 0$ is a constant from Korn's inequality (independant of $h$).}
By applying \textbf{Corollary 4.21, 4.27}, \textbf{Theorem 6.2} from \cite{da_veiga_mathematical_2014} and Aubin-Nitsche theorem, we can get the following estimates : 
\begin{equation}
\begin{aligned}
& \norm{\vec u - \vec u^h}_{\boldsymbol H^1(\Omega)} \lesssim \frac{3\lambda + 2\mu}{2\mu C} h^q \norm{\vec u}_{\boldsymbol H^{q+1}(\Omega)} \\
& \norm{\vec u - \vec u^h}_{\boldsymbol L^2(\Omega)} \lesssim (3\lambda + 2\mu) h^{q+1}\norm{\vec u}_{\boldsymbol H^{q+1}(\Omega)}
\end{aligned}
\label{eq:convergence_estimates}
\end{equation}
\textit{where $q = \min{(s,d)}$}. In this case, $u_e$ is an infinitly smooth solution, then $q = d$. 

\begin{figure}[!h]
	\centering
	\begin{subfigure}[b]{0.49\textwidth}
		\centering
		\includegraphics[width=\textwidth]{figures_non_mixed_DH/convergence_plot_degree_2_lambda=1.25.png}
		\caption{$d=2$}
		\label{fig:deg2_NMDHBC}
	\end{subfigure}
	\begin{subfigure}[b]{0.49\textwidth}
		\centering
		\includegraphics[width=\textwidth]{figures_non_mixed_DH/convergence_plot_degree_3_lambda=1.25.png}
		\caption{$d=3$}
		\label{fig:deg3_NMDHBC}
	\end{subfigure}
	\\
	\begin{subfigure}[b]{0.49\textwidth}
		\centering
		\includegraphics[width=\textwidth]{figures_non_mixed_DH/convergence_plot_degree_4_lambda=1.25.png}
		\caption{$d=4$}
		\label{fig:deg4_NMDHBC}
	\end{subfigure}
	\begin{subfigure}[b]{0.49\textwidth}
		\centering
		\includegraphics[width=\textwidth]{figures_non_mixed_DH/convergence_plot_degree_5_lambda=1.25.png}
		\caption{$d=5$}
		\label{fig:deg5_NMDHBC}
	\end{subfigure}
	\caption{Four plots of relative errors ($\norm{\vec u - \vec u^h}_{\boldsymbol H^1(\Omega)}$ and $\norm{\vec u - \vec u^h}_{\boldsymbol L^2(\Omega)}$) with different degree B-Splines (with $\lambda = 1.25$ and $\mu = 1$)}
	\label{fig:four_errors_graphs_1}
\end{figure}

We can see that the numerical solution converges with the expected order to the real solution : the more the mesh is refined, the more the error is lower and the more the splines' degree is higher, the more the convergence rate is higher. 
For this numerical resolution, I have used \textbf{GMRES} method to solve the linear system at the end of the discretization. That needed to take care of the tolerance of the solver, because with a fixed tolerance, if the $d$ and $n_{\text{cells}}$ are "too higher", the convergence rate cannot hold. The following figure illustrates this case, when $d=5$ and $n_{\text{cells}} = 16$.

\begin{figure}[!h]
	\centering
	\includegraphics[width=0.5\linewidth]{degre5_nul}
	\caption{Error plot between exact and numerical solution with degree 5 B-Splines with "higher" tolerance for the solvera}
\end{figure}

But even with a lower tolerance for the solver ($10^{-15}$ for figure \ref{fig:deg5_NMDHBC}), the error reaches a plateau, which is not the case for lower degree B-Splines (at least for the tested configurations). This plateau can be explained by the matrix of the bilinear form $a(.,.)$ which is ill-conditioned. Moreover, the bad conditioning can become worse with more refined mesh and higher degree B-Splines (as the convergence is faster).

\subsection{Mixed boundary conditions}
\label{sec:mixed_BC}

Then, I have made a simulation with Mixed Boundary Conditions, as it is introduced in \eqref{eq:strong_form}, with also non-homogeneous Dirichlet boundary conditions.

Again, $\Omega = [0,1]^3$, with : 
\begin{equation*}
	\begin{aligned}
		& \partial \Omega_{\text{DH}} = \left\{  (x,y,z) \in \partial\Omega \left| z = 0 \text{ or } z = 1 \right.  \right\} \hspace{0.5cm} 
		\partial \Omega_{\text{DNH}} = \left\{  (x,y,z) \in \partial\Omega \left| x = 0 \text{ or } x = 1 \right.  \right\} \\
		& \partial \Omega_{\text{N}} = \left\{  (x,y,z) \in \partial\Omega \left| y = 0 \text{ or } y = 1 \right.  \right\}
	\end{aligned}
\end{equation*}
For that purpose, $\vec u_e, \vec f, \vec t_N$ and $\vec g$ are now defined as : 
\begin{equation*}
\begin{aligned}
	& \vec u_e =
	\begin{pmatrix}
		0 \\
		0 \\
		\sin{\left(\pi z \right)} \cos{\left(\pi x \right)} \cos{\left(\pi y \right)}
	\end{pmatrix}, \hfill \vec f =  \begin{pmatrix}
	\pi^{2} \left(\lambda + \mu\right) \sin{\left(\pi x \right)} \cos{\left(\pi y \right)} \cos{\left(\pi z \right)} \\
	\pi^{2} \left(\lambda + \mu\right) \sin{\left(\pi y \right)} \cos{\left(\pi x \right)} \cos{\left(\pi z \right)} \\
	\pi^{2} \left(\lambda + 4 \mu\right) \sin{\left(\pi z \right)} \cos{\left(\pi x \right)} \cos{\left(\pi y \right)}
	\end{pmatrix} \\
	& \vec t_N = \begin{pmatrix}
	0 \\
	- \lambda \pi \cos{\left(\pi x \right)} \cos{\left(\pi z \right)} \\
	0
	\end{pmatrix}, \hfill \vec g = \begin{pmatrix}
	0 \\
	0 \\
	\pm \sin{\left(\pi z \right)} \cos{\left(\pi y \right)}
	\end{pmatrix}
\end{aligned}
\end{equation*} 	

Then, we can get the same plots and interpretation as before. 

\begin{figure}[!h]
	\centering
	\includegraphics[width=0.8\linewidth]{3d_plots_degree_2_non_mixed_ncell=128_function_u3_planes_z.png}
	\caption{Plots comparison between the simulated and real $u_{e,3}$ with the error, for different plane $z$, with $d=2$ and $n_{\text{cell}} = 128$}
	\label{fig:3D_plots_mixed_BC}
\end{figure}

\begin{figure}[!h]
	\centering
	\begin{subfigure}[b]{0.49\textwidth}
		\centering
		\includegraphics[width=\textwidth]{figures_non_mixed/convergence_plot_degree_2_lambda=1.25.png}
		\caption{$d=2$}
		\label{fig:deg2_NM}
	\end{subfigure}
	\begin{subfigure}[b]{0.49\textwidth}
		\centering
		\includegraphics[width=\textwidth]{figures_non_mixed/convergence_plot_degree_3_lambda=1.25.png}
		\caption{$d=3$}
		\label{fig:deg3_NM}
	\end{subfigure}
	\begin{subfigure}[b]{0.49\textwidth}
		\centering
		\includegraphics[width=\textwidth]{figures_non_mixed/convergence_plot_degree_4_lambda=1.25.png}
		\caption{$d=4$}
		\label{fig:deg4_NM}
	\end{subfigure}
	\begin{subfigure}[b]{0.49\textwidth}
		\centering
		\includegraphics[width=\textwidth]{figures_non_mixed/convergence_plot_degree_5_lambda=1.25.png}
		\caption{$d=5$}
		\label{fig:deg5_NM}
	\end{subfigure}
	\caption{Four plots of relative errors ($\norm{\vec u - \vec u^h}_{\boldsymbol H^1(\Omega)}$ and $\norm{\vec u - \vec u^h}_{\boldsymbol L^2(\Omega)}$) with different degree B-Splines (with $\lambda = 1.25$ and $\mu = 1$)}
	\label{fig:four_errors_graphs}
\end{figure}

\newpage
For fixed value $\lambda = 1.25$ and $\mu = 1$, the codes seems to give a numerical solution that converges to the theoritical one with the expected order. We can notice that the plateau effect comes back and appears faster than in the case of Dirichlet Homogeneous Boundary Conditions. The $L^2$ error seems to converge to a value slightly higher than $10^{-8}$, while the $H^1$ error converges to a value slightly higher than $10^{-5}$, and this phenomena is more pronounced for higher degree B-Splines. The limit value of each error seems to be the same for all degree B-Splines, when the $L^2$ error becomes lower than $10^{-7}$ and the $H^1$ error becomes lower than $10^{-5}$. So, this plateau effect is not due to the degree of B-Splines, but rather to the ill-conditionedness of the matrix of the bilinear form $a(.,.)$. This ill-conditionedness becomes worse with the refinement of the mesh and the increase of the degree of B-Splines. A way to verify this hypothesis is to solve the linear system with a different solver, with a preconditioner applyied to the matrix before the resolution.

These simulations have been made with $\lambda = 1.25$ and $\mu = 1$. We can now look at them for more realistic values of $\lambda$ and $\mu$. The following table \ref{tab:lame_coefficients} gives the Lamé coefficients for different materials. 
\begin{table}[h]
\centering
\begin{tabular}{|c|c|c|}
\hline
Material & $\lambda$ (GPa) & $\mu$ (GPa) \\
\hline
Steel & 120 & 80 \\
\hline
Concrete & 17 & 14 \\
\hline
Rubber & 0.16 & 0.00033 \\
\hline
\end{tabular}
\caption{Lamé coefficients for different materials}
\label{tab:lame_coefficients}
\end{table}

\begin{figure}[!h]
	\centering
	\begin{subfigure}[b]{0.49\textwidth}
		\centering
		\includegraphics[width=\textwidth]{figures_non_mixed_realistic/convergence_plot_degree_2_lambda=160000000.0.png}
		\caption{Rubber with degree 2 B-Splines}
		\label{fig:rubber_degree_2}
	\end{subfigure}
	\begin{subfigure}[b]{0.49\textwidth}
		\centering
		\includegraphics[width=\textwidth]{figures_non_mixed_realistic/convergence_plot_degree_4_lambda=160000000.0.png}
		\caption{Rubber with degree 4 B-Splines}
		\label{fig:rubber_degree_4}
	\end{subfigure}
	\centering
	\begin{subfigure}[b]{0.49\textwidth}
		\centering
		\includegraphics[width=\textwidth]{figures_non_mixed_realistic/convergence_plot_degree_2_lambda=17000000000.0.png}
		\caption{Concrete with degree 2 B-Splines}
		\label{fig:concrete_degree_2}
	\end{subfigure}
	\begin{subfigure}[b]{0.49\textwidth}
		\centering
		\includegraphics[width=\textwidth]{figures_non_mixed_realistic/convergence_plot_degree_4_lambda=17000000000.0.png}
		\caption{Concrete with degree 4 B-Splines}
		\label{fig:concrete_degree_4}
	\end{subfigure}
	\centering
	\begin{subfigure}[b]{0.49\textwidth}
		\centering
		\includegraphics[width=\textwidth]{figures_non_mixed_realistic/convergence_plot_degree_2_lambda=120000000000.0.png}
		\caption{Steel with degree 2 B-Splines}
		\label{fig:steel_degree_2}
	\end{subfigure}
	\begin{subfigure}[b]{0.49\textwidth}
		\centering
		\includegraphics[width=\textwidth]{figures_non_mixed_realistic/convergence_plot_degree_4_lambda=120000000000.0.png}
		\caption{Steel with degree 4 B-Splines}
		\label{fig:steel_degree_4}
	\end{subfigure}
	\caption{Error plots between exact solution and numerical solution with realistic values of Lamés' coefficients with degree 2 and 4 B-Splines}
	\label{fig:error_realistic_lame}
\end{figure}

As we can see in figure \ref{fig:error_realistic_lame}, in the case of concrete and steel, the errors behaves as expected : well convergence with expected order and the plateau effect appears after a certain value of the error. 
But for the rubber, the convergence order is not really respected, the error decreases faster than expected and the plateau degenerates... 
To explain this, we can look at the estimates \eqref{eq:convergence_estimates} :
$\norm{\vec u - \vec u^h}_{\boldsymbol H^1(\Omega)} \lesssim \frac{3\lambda + 2\mu}{2\mu C} h^d \norm{\vec u}_{\boldsymbol H^{d+1}(\Omega)}$
As $\lambda$ is very high compared to $\mu$, the estimates are not really valid anymore, as $\frac{3\lambda + 2\mu}{2\mu C} \rightarrow + \infty$ when $\lambda \rightarrow + \infty$. 
This is the case for incompressible materials, like rubber, where $\lambda$ is very high compared to $\mu$. To get a better idea of the convergence, we can look at the error plots for different values of $\lambda$ with $\mu = 1$.

\begin{figure}[!h]
	\centering
	\begin{subfigure}[b]{0.49\textwidth}
		\centering
		\includegraphics[width=\textwidth]{figures_increasing_lambda_non_mixed/convergence_plot_degree_2_lambda=1.25.png}
		\caption{$\lambda=1.25$~Pa}
	\end{subfigure}
	\begin{subfigure}[b]{0.49\textwidth}
		\centering
		\includegraphics[width=\textwidth]{figures_increasing_lambda_non_mixed/convergence_plot_degree_2_lambda=10.0.png}
		\caption{$\lambda=10$~Pa}
	\end{subfigure}
	\centering
	\begin{subfigure}[b]{0.49\textwidth}
		\centering
		\includegraphics[width=\textwidth]{figures_increasing_lambda_non_mixed/convergence_plot_degree_2_lambda=100.0.png}
		\caption{$\lambda=100$~Pa}
	\end{subfigure}
	\begin{subfigure}[b]{0.49\textwidth}
		\centering
		\includegraphics[width=\textwidth]{figures_increasing_lambda_non_mixed/convergence_plot_degree_2_lambda=1000.0.png}
		\caption{$\lambda=1000$~Pa}
	\end{subfigure}
	\\
	\begin{subfigure}[b]{0.49\textwidth}
		\centering
		\includegraphics[width=\textwidth]{figures_increasing_lambda_non_mixed/convergence_plot_degree_2_lambda=10000.0.png}
		\caption{$\lambda=10000$~Pa}
	\end{subfigure}
	\begin{subfigure}[b]{0.49\textwidth}
		\centering
		\includegraphics[width=\textwidth]{figures_increasing_lambda_non_mixed/convergence_plot_degree_2_lambda=100000.0.png}
		\caption{$\lambda=100000$~Pa}
	\end{subfigure}
	\caption{Error plots between exact solution and numerical solution with degree 2 B-Splines with increasing values of $\lambda$ with $\mu = 1$~Pa}
	\label{fig:error_kappa}
\end{figure}

As we can see in figure \ref{fig:error_kappa}, the convergence order is not respected anymore when $\lambda$ is too high compared to $\mu$. We can observe that when $\frac{\lambda}{\mu} = 10^4$, the error behaves like the error curves for the rubber with degree 2 B-Splines in figure \ref{fig:rubber_degree_2} $\displaystyle \left(\frac{\lambda_{\text{rubber}}}{\mu_{\text{rubber}}} \approx 500\right)$. Then, for values of $\lambda$ such that $\frac{\lambda}{\mu} \geq 10^4$, the error is not even decreasing anymore.

Two hypothesis can be made to explain this phenomena : 
\begin{itemize}
	\item The estimates \eqref{eq:convergence_estimates} are not valid anymore, as $\lambda$ is too high compared to $\mu$.
	\item The bilinear form $a(.,.)$ is ill-conditioned, leading to numerical instability when $\lambda$ is "too high" compared to $\mu$.
\end{itemize}

One can try to increase the degree of B-Splines to see if the convergence is reinforced as the power of $h$ in the estimates is higher. 

\begin{figure}
	\centering
	\includegraphics[width=0.8\textwidth]{figures_increasing_lambda_non_mixed/convergence_plot_degree_4_lambda=100000.0.png}	
	\caption{Error plot between exact solution and numerical solution with degree 4 B-Splines with $\lambda = 100000$~Pa and $\mu = 1$~Pa}
	\label{fig:error_kappa_degree_4}
\end{figure}

As we can see in figure \ref{fig:error_kappa_degree_4}, the convergence is not reinforced, the error is still not decreasing. Actually, this phenomena is called \textbf{locking} and it is a well-known issue in FEM for incompressible materials. The locking occurs when the numerical solution is too stiff and does not converge to the real solution for high values of $\lambda$ compared to $\mu$.
One way to avoid this issue is to use a mixed formulation, where the pressure field is introduced as an additional variable. This will be presented in the next section.

\section{Mixed displacement-pressure formulation}
To introduce this formulation, let's consider again the strong formulation from \eqref{eq:strong_form}. Then, denotes by $p$ the pressure field, defined by $\displaystyle p = - \lambda \div{\vec u}$. As we are seeking $\vec u \in \boldsymbol{H}^1(\Omega)$, $p$ is naturally living in $L^2(\Omega)$. Then Hook's law can be rewritten : $\boldsymbol{\sigma} (\vec u,p) = -p I_3 + 2\mu \boldsymbol{\varepsilon(\vec u)} \text{ in }  \Omega $ \\
So the strong formulation can be re-write as : For a given $\vec f \in \boldsymbol L^2(\Omega)$, $\vec t_N \in \boldsymbol L^2(\partial \Omega_N)$, find $(\vec u, p) \in \boldsymbol H^1(\Omega)\times L^2(\Omega)$ such that : 

\begin{equation}
	\label{eq:strong_pressure_disp_form}
	\left\{
	\begin{aligned}
		& - \div \boldsymbol{\sigma} (\vec u,p) = \vec f & \text{ in } & \Omega \hspace{1cm} \text{(Equilibrium)}\\
		& \div{\vec u} + \frac{1}{\lambda} p = 0 & \text{ in } & \Omega \hspace{1cm} \text{(Pressure field definition)} \\
		& \vec u = \vec 0 & \text{ in } &\partial \Omega_D \\
		& \boldsymbol{\sigma} (\vec u) \cdot \vec n = \vec t_N & \text{ in }& \partial \Omega_N
	\end{aligned}
	\right.
\end{equation}

Now, to derive the weak formulation associated, first we multiply the Equilibrium equation by $v \in \boldsymbol H^1_{\vec 0,D}(\Omega)$ and we interger by part on $\Omega$, then we multiply the Pressure field equation by $q \in L^2(\Omega)$ and we interger over $\Omega$. Then we just have to add both integrals. 
Then, the weak formulation is: 
\begin{equation}
	\label{eq:weak_pressure_displ_form}
	\boxed{
	\left\{
	\begin{aligned}
		&\text{Find } (\vec u,p) \in \boldsymbol H^1_{\vec 0,D}(\Omega) \times L^2(\Omega) \text{ such that :}\\
		& \tilde{a}((\vec u,p),(\vec v,q)) = l(\vec v,q) \hspace{0.5cm} \forall (\vec v,q) \in \boldsymbol H^1_{\vec 0,D}(\Omega) \times L^2(\Omega)
	\end{aligned}
	\right.
	}
\end{equation}

With : 
\begin{equation*}
	\begin{aligned}
		& \tilde{a} : \left\{
		\begin{aligned}
			&\left( \boldsymbol H^1_{\vec 0,D}(\Omega) \times L^2(\Omega) \right)^2 \rightarrow \mtr \\
			&((\vec u,p),(\vec v,q))  \longmapsto \int_\Omega \left( 2\mu \boldsymbol{\varepsilon}(\vec u) : \boldsymbol{\varepsilon}(\vec v) - p (\div \vec v) + (\div \vec u)q + \frac{1}{\lambda} pq \right) \dif V
		\end{aligned}
		\right. \\
		& l : 
		\left\{
		\begin{aligned}
			&\boldsymbol H^1_{\vec 0,D}(\Omega) \rightarrow \mtr \\
			&\vec v \longmapsto \int_\Omega \vec f . \vec v \dif V + \int_{\partial\Omega_N} \vec t_N.\vec v \dif S
		\end{aligned}
		\right.
	\end{aligned}
\end{equation*}


\subsection{Well-posedness of the mixed displacement–pressure formulation}

Let
\[
V:=\boldsymbol H^1_{\vec 0,D}(\Omega),\qquad Q:=L^2(\Omega),\qquad X:=V\times Q,
\]
with the product norm
\[
\|(\vec u,p)\|_X^2 := \|\vec u\|_{\boldsymbol H^1(\Omega)}^2 + \|p\|_{L^2(\Omega)}^2.
\]

\noindent \textbf{Now, we can prove the well-posedness of \eqref{eq:weak_pressure_displ_form} for every finite }$\lambda>0, \mu>0$ \text{ and } $\Omega_D \neq \emptyset$.
\begin{itemize}
	\item $(X,\|\cdot\|_X)$ is a Hilbert space.
	\item $\tilde a(\cdot,\cdot)$ is a bilinear form on $X\times X$ and $l(\cdot)$ is linear on $X$.
	\item $l(\cdot)$ is continuous: for all $(\vec v,q)\in X$,
	\begin{equation*}
		\begin{aligned}
			|l(\vec v,q)|
			&\le |(\vec f,\vec v)_{\boldsymbol L^2(\Omega)}| + |(\vec t_N,\vec v)_{\boldsymbol L^2(\partial\Omega_N)}| \\
			&\le \Big(\|\vec f\|_{\boldsymbol L^2(\Omega)} + C_{\text{tr}}\|\vec t_N\|_{\boldsymbol L^2(\partial\Omega_N)}\Big)\|\vec v\|_{H^1(\Omega)} \\
			|l(\vec v,q)| &\le C \big(\|\vec f\|_{\boldsymbol L^2(\Omega)} + \|\vec t_N\|_{\boldsymbol L^2(\partial\Omega_N)}\big)\|(\vec v,q)\|_X,
		\end{aligned}
	\end{equation*}
	where $C_{\text{tr}}$ is the trace constant and $C$ a generic constant.
	\item $\tilde a(\cdot,\cdot)$ is continuous: for all $(\vec u,p),(\vec v,q)\in X$,
	\begin{equation*}
		\begin{aligned}
			|\tilde a((\vec u,p),(\vec v,q))|
			&\le 2\mu \|\boldsymbol{\varepsilon}(\vec u)\|_{L^2}\|\boldsymbol{\varepsilon}(\vec v)\|_{L^2}
				+ \|p\|_{L^2}\|\div \vec v\|_{L^2}
				+ \|\div \vec u\|_{L^2}\|q\|_{L^2}
				+ \tfrac{1}{\lambda}\|p\|_{L^2}\|q\|_{L^2} \\
			&\le 2 \mu \|\vec u\|_{\boldsymbol H^1}\|\vec v\|_{\boldsymbol H^1}
				+ 3 \|p\|_{L^2}\|\vec v\|_{\boldsymbol H^1}
				+ 3 \|\vec u\|_{\boldsymbol H^1}\|q\|_{L^2}
				+ \frac{1}{\lambda}\|p\|_{L^2}\|q\|_{L^2} \\
			&\le 2 \mu \| (\vec u,p)\|_X \|(\vec v,q)\|_X
				+ 3 \|(\vec u,p)\|_X \|(\vec v,q)\|_X
				+ 3 \|(\vec u,p)\|_X \|(\vec v,q)\|_X \\
				& + \frac{1}{\lambda}\|(\vec u,p)\|_X \|(\vec v,q)\|_X \\
			|\tilde a((\vec u,p),(\vec v,q))| &\le (2\mu + 6 + \frac{1}{\lambda})\|(\vec u,p)\|_X \|(\vec v,q)\|_X.
		\end{aligned}
	\end{equation*}
	using $\|\div \vec w\|_{L^2}\le 3 \|\vec w\|_{H^1}$, $\|p\|_{L^2} \le \|(\vec u,p)\|_X$ and $\|\vec u\|_{H^1} \le \|(\vec u,q)\|_X$.
	\item $\tilde a(\cdot,\cdot)$ is coercive on $X$: for all $(\vec u,p)\in X$,
	\begin{equation*}
		\tilde a((\vec u,p),(\vec u,p)) = 2\mu \|\boldsymbol{\varepsilon}(\vec u)\|_{L^2}^2 + \tfrac{1}{\lambda}\|p\|_{L^2}^2
		\ge 2\mu C_K^2 \|\vec u\|_{H^1}^2 + \tfrac{1}{\lambda}\|p\|_{L^2}^2
		\ge \alpha \|(\vec u,p)\|_X^2,
	\end{equation*}
	with $\alpha:=\min\{2\mu C_K^2,\,1/\lambda\}>0$ (for fixed finite $\lambda$).
\end{itemize}

Then, by Lax–Milgram theorem, the mixed problem \eqref{eq:weak_pressure_displ_form} admits a unique solution $(\vec u,p)\in X$ and the stability estimate
\[
\|\vec u\|_{H^1(\Omega)}+\|p\|_{L^2(\Omega)} \le C\big(\|\vec f\|_{\boldsymbol L^2(\Omega)} + \|\vec t_N\|_{\boldsymbol L^2(\partial\Omega_N)}\big)
\]
holds with $C$ independent of $(\vec f,\vec t_N)$.

\noindent \textbf{Remark.} As $\lambda\to \infty$ (near-incompressible limit) the coercivity constant degenerates on the pressure component; the formulation approaches a saddle-point (Stokes-like) problem requiring an inf-sup (Ladyzhenskaya–Babuška–Brezzi) condition on the pair $(V,Q)$ to remain stable.

A way to ensure the inf-sup condition is to impose a non-zero mean value for the pressure field, i.e., $p\in L^2_0(\Omega)$, which is a common practice in numerical simulations of incompressible flows.

Then, the following inf-sup condition holds:
\[\inf_{q\in Q}\sup_{\vec v\in V}\frac{\tilde a((\vec v,q),(\vec u,p))}{\|\vec v\|_{H^1(\Omega)}\|q\|_{L^2(\Omega)}} \ge \beta > 0,\]
for some $\beta>0$ independent of $\lambda$ and $\mu$. This condition ensures the stability of the mixed formulation and prevents spurious oscillations in the pressure field, and V = $H^1_0(\Omega)$ and $Q = L^2_0(\Omega)$ \cite{boffi_mixed_2008}.

If $\Omega_D = \emptyset$, the Dirichlet boundary condition is not imposed, and the problem becomes a pure Stokes problem. In this case, the inf-sup condition is necessary to ensure the well-posedness of the problem, and the uniqueness of the solution is guaranteed by the compatibility condition on the external loads, as discussed in Section \ref{sec:pure_traction} modulo a rigid body motion.

\subsection{Isogeometric discretisation and a priori error estimate}
Under standard approximation and stability assumptions for the mixed formulation, the Galerkin solution enjoys the usual quasi-optimal a priori estimate. Let $(V_h,Q_h)\subset V\times Q$ be the tensor–product B-spline spaces of (vector) degree $d$ (for the displacement) and (scalar) degree $d$ or $d-1$ (for the pressure) with mesh size $h$, and assume that the discrete inf–sup condition holds with a constant $\beta>0$ independent of $h$, $\lambda$ and $\mu$ (locking–free setting) \cite{boffi_mixed_2008}. Then, for the exact solution $(\vec u,p)$ of \eqref{eq:weak_pressure_displ_form} and the discrete solution $(\vec u_h,p_h)\in V_h\times Q_h$, continuity plus coercivity on the kernel give the Céa type bound
\[
\|\vec u-\vec u_h\|_{\boldsymbol H^1(\Omega)}+\|p-p_h\|_{L^2(\Omega)}
\;\le\; C\,\Big( \inf_{\vec w_h\in V_h}\|\vec u-\vec w_h\|_{\boldsymbol H^1(\Omega)} + \inf_{q_h\in Q_h}\|p-q_h\|_{L^2(\Omega)}\Big),
\]
with $C$ independent of $h,\lambda,\mu$. For $(\vec u,p)$ sufficiently smooth, namely $\vec u\in \boldsymbol H^{d+1}(\Omega)$ and $p\in H^{d}(\Omega)$, spline approximation estimates yield
\[
\|\vec u-\vec u_h\|_{\boldsymbol H^1(\Omega)}+\|p-p_h\|_{L^2(\Omega)} \;\le\; C h^{d}\Big(\|\vec u\|_{\boldsymbol H^{d+1}(\Omega)}+\|p\|_{H^{d}(\Omega)}\Big),
\]
and, under the additional elliptic regularity needed for a duality argument, the displacement $L^2$–error improves to
\[
\|\vec u-\vec u_h\|_{\boldsymbol L^2(\Omega)} \;\le\; C h^{d+1}\|\vec u\|_{\boldsymbol H^{d+1}(\Omega)}.
\]
Crucially, these constants remain bounded as $\lambda\to\infty$, in contrast with the pure displacement formulation whose coercivity deteriorates ($\alpha\sim 1/\lambda$), leading to volumetric locking; hence the mixed formulation delivers uniform (locking–free) convergence in the nearly incompressible regime \cite{boffi_mixed_2008,da_veiga_mathematical_2014}.






% Let  and  be tensor-product B-spline spaces of degree  and mesh-size  (maximal continuity). If the exact solution satisfies the regularity
% , standard approximation results for B-splines give the best-approximation estimates
% \begin{equation*}
% \inf_{(\vec w_h,q_h)\in V_h\times Q_h} \bigl(|\vec u-\vec w_h|{H^1} + |p-q_h|{L^2}\bigr) = O(h^d).
% \end{equation*}
% By C'ea's lemma (Galerkin orthogonality + continuity/coercivity) the discrete solution  satisfies
% \begin{equation}\label{eq:cea}
% |\vec u-\vec u_h|{H^1} + |p-p_h|{L^2} \le \frac{M}{\alpha};\inf_{(\vec w_h,q_h)}\bigl(|\vec u-\vec w_h|{H^1} + |p-q_h|{L^2}\bigr) = O(h^d),
% \end{equation}
% where the constants  are the continuity and coercivity constants. In practice one thus expects order  convergence in the -norm for displacements and in the -norm for pressure provided the solution is smooth enough.



\subsection{Example of simulation using \texttt{PSYDAC}}
In this section, we will present the results of a simulation using the mixed displacement-pressure formulation with B-Splines. 

For the manufactured solution, we will use the same one as in section \ref{subsec:Dirichlet Homogeneous Boundary conditions on}. For reminder, the exact solution is given by:
$$
\vec u_e(x,y,z)=
\begin{pmatrix}
0\\[2pt]
0\\[2pt]
\sin(\pi x)\sin(\pi y)\sin(\pi z)
\end{pmatrix},\qquad
p_e(x,y,z)= -\lambda\,\div \vec u_e = -\lambda \pi \sin(\pi x)\sin(\pi y)\cos(\pi z).
$$

$$
\vec f = -\div \boldsymbol\sigma(\vec u_e,p_e)=
\begin{pmatrix}
-\pi^{2}(\lambda+\mu)\cos(\pi x)\sin(\pi y)\cos(\pi z)\\[4pt]
-\pi^{2}(\lambda+\mu)\sin(\pi x)\cos(\pi y)\cos(\pi z)\\[4pt]
\pi^{2}(\lambda+4\mu)\sin(\pi x)\sin(\pi y)\sin(\pi z)
\end{pmatrix}.
% -\pi^{2}(\lambda+\mu)\cos(\pi x)\sin(\pi y)\cos(\pi z)\[4pt]\\n-\pi^{2}(\lambda+\mu)\sin(\pi x)\cos(\pi y)\cos(\pi z)\[4pt]\\n\pi^{2}(\lambda+4\mu)\sin(\pi x)\sin(\pi y)\sin(\pi z)
% \end{pmatrix}.
$$


\begin{figure}[!h]
	\centering
	\begin{subfigure}[b]{0.49\textwidth}
		\centering
		\includegraphics[width=\textwidth]{convergence_degree_2_mixed_dirichlet_homogeneous_kappa=1.25}
		\caption{$d=2$}
		\label{fig:convergencedegree2mixeddirichlethomogeneouskappa1}
	\end{subfigure}
	\begin{subfigure}[b]{0.49\textwidth}
		\centering
		\includegraphics[width=\textwidth]{convergence_degree_3_mixed_dirichlet_homogeneous_kappa=1.25}
		\caption{$d=3$}
		\label{fig:convergencedegree3mixeddirichlethomogeneouskappa1}
	\end{subfigure}
	\\
	\begin{subfigure}[b]{0.49\textwidth}
		\centering
		\includegraphics[width=\textwidth]{convergence_degree_4_mixed_dirichlet_homogeneous_kappa=1.25}
		\caption{$d=4$}
		\label{fig:convergencedegree4mixeddirichlethomogeneouskappa1}
	\end{subfigure}
	\begin{subfigure}[b]{0.49\textwidth}
		\centering
		\includegraphics[width=\textwidth]{convergence_degree_5_mixed_dirichlet_homogeneous_kappa=1.25}
		\caption{$d=5$}
		\label{fig:convergencedegree5mixeddirichlethomogeneouskappa1}
	\end{subfigure}
	\caption{Four plots of errors between exact solution and numerical solution with different degree B-Splines (with $\lambda = 1.25$ and $\mu = 1$) using the mixed formulation}
\end{figure}

\begin{figure}[!h]
	\centering
	\begin{subfigure}[b]{0.49\textwidth}
		\centering
		\includegraphics[width=\textwidth]{convergence_degree_2_mixed_dirichlet_homogeneous_kappa=1.25}
		\caption{$\lambda=1.25$}
	\end{subfigure}
	\begin{subfigure}[b]{0.49\textwidth}
		\centering
		\includegraphics[width=\textwidth]{convergence_degree_2_mixed_dirichlet_homogeneous_kappa=10.0}
		\caption{$\lambda=10$}
	\end{subfigure}
	\\
	\begin{subfigure}[b]{0.49\textwidth}
		\centering
		\includegraphics[width=\textwidth]{convergence_degree_2_mixed_dirichlet_homogeneous_kappa=100.0}
		\caption{$\lambda=100$}
	\end{subfigure}
	\begin{subfigure}[b]{0.49\textwidth}
		\centering
		\includegraphics[width=\textwidth]{convergence_degree_2_mixed_dirichlet_homogeneous_kappa=1000.0}
		\caption{$\lambda=1000$}
	\end{subfigure}
	\\
	\begin{subfigure}[b]{0.49\textwidth}
		\centering
		\includegraphics[width=\textwidth]{convergence_degree_2_mixed_dirichlet_homogeneous_kappa=10000.0}
		\caption{$\lambda=10000$}
	\end{subfigure}
	\begin{subfigure}[b]{0.49\textwidth}
		\centering
		\includegraphics[width=\textwidth]{convergence_degree_2_mixed_dirichlet_homogeneous_kappa=100000.0}
		\caption{$\lambda=100000$}
	\end{subfigure}
	\caption{Error plots between exact solution and numerical solution with degree 2 B-Splines and mixed formulation with increasing values of $\lambda$}
\end{figure}

\chapter{Thermoelasticity}
\section{Introduction}
This chapter is dedicated to the study of the thermoelasticity problem, which is a coupled problem between the mechanical and thermal fields. The aim is to solve the coupled system of equations that govern the behavior of a thermoelastic material under thermal and mechanical loads. This work has been initiated to solve a problem present during the IPP Programme Days 2025 by Prof. Dr. Rudolf Neu. The goal is to understand the deformation of a medium under the influence of thermal loads and mechanical forces. The medium is composite material, composed of copper and tungsten. The geometry is presented on the following figure.
\begin{figure}[!h]
	\centering
	\includegraphics[width=0.7\textwidth]{geometry_thermoelasticity.png}
	\caption{Geometry of the thermoelasticity problem from the IPP Programme Days 2025}
	\label{fig:geometry_thermoelasticity}
\end{figure}

\section{Simplified problem - Physical model and equations}

Before solving this problem, I want to simplify it a bit by considering a slightly different geometry, which is a square without the hole in the middle. The goal is to understand the behavior of the medium under thermal and mechanical loads, to see how the deformation of the medium is affected by the thermal loads and to see if PSYDAC can effectively solve this simplified problem. The geometry of the simplified problem is presented on the figure \ref{fig:geometry_thermoelasticity_simplified}.

\subsection{Problem setup and parameters}
\label{subsec:problem_setup_parameters}
The problem can be parameterized as follows with other boundary conditions and parameters:
\begin{itemize}
	\item The domain $\Omega$ is a square of size $22$: $\Omega = [0,22]^2$.
	\item $\partial \Omega_{\text{heat}} = \{(x,y) \in \partial \Omega \,|\, x = 0 \text{ and } x = 22 \}$ is the left and right edges of the square, where heat $Q$ is applied. $\partial \Omega_{\text{temp}} = \{(x,y) \in \partial \Omega \,|\, y = 0 \text{ and } y = 22\}$ is the top and bottom edges of the square, where the temperature is fixed to $T_0$.
	\item $\partial \Omega_{D} = \{(x,y) \in \partial \Omega \,|\, x = 0 \text{ and } x = 22 \}$ is the left and right edges of the square, where the Dirichlet boundary conditions are applied and the displacement is fixed. $\partial \Omega_{N} = \{(x,y) \in \partial \Omega \,|\, y = 0 \text{ and } y = 22\}$ is the top and bottom edges of the square, where the Neumann boundary conditions are applied and the traction is fixed.
	\item For $y \in (0, 17)$, the material is composite and properties are given by a affine interpolation between the properties of copper and tungsten. For $y \in (17, 22)$, the material is pure tungsten.
	For example, $\lambda$ is defined as:
	\[\lambda(y) = \begin{cases}
		\lambda_{\text{Cu}} + \frac{y}{17}(\lambda_{\text{W}} - \lambda_{\text{Cu}}) & \text{if } y \in (0, 17) \\
		\lambda_{\text{W}} & \text{if } y \in (17, 22)
	\end{cases}\]
	\item The material properties are given by: 
	\begin{table}[h!]
	\centering
	\begin{tabular}{lcc}
	\hline
	\textbf{Property} & \textbf{Cu} & \textbf{W} \\
	\hline
	Lamé's first constant $\lambda$ [Pa] & $1.1\times 10^{11}$ & $2.5\times 10^{11}$ \\
	Shear modulus $\mu$ [Pa]             & $4.1\times 10^{10}$ & $1.6\times 10^{11}$ \\
	Thermal expansion $\alpha$ [1/K]     & $1.7\times 10^{-5}$ & $4.5\times 10^{-6}$ \\
	Thermal conductivity $\kappa$ [W/(m$\cdot$K)] & $401$ & $174$ \\
	\hline
	\end{tabular}
	\caption{Material properties of Copper (Cu) and Tungsten (W).}
	\end{table}
\end{itemize}

\begin{figure}[!h]
	\centering
	\includegraphics[width=0.6\textwidth]{geometry_thermoelasticity_simplified.jpg}
	\caption{Geometry of the simplified thermoelasticity problem}
	\label{fig:geometry_thermoelasticity_simplified}
\end{figure}

\subsection{Strong Formulation}
The presentation of the strong formulation of the thermoelasticity problem is based on the principles of continuum mechanics and thermodynamics taken from \cite{MeyerSpringer2015} and \cite{Persson2016}. The thermoelasticity problem is a coupled problem between the mechanical and thermal fields. The equations that govern the behavior of a thermoelastic material are given by the coupling of the heat equation and the elasticity equation.

The full time-dependent thermoelasticity equations are:
\begin{equation}
	\left\{
	\begin{aligned}
		\rho(y) c_p(y) \frac{\partial T}{\partial t} + \div (\kappa(y) \grad T) &= Q - (3\lambda(y) + 2\mu(y))\alpha(y) T_0 \frac{\partial}{\partial t} (\div \vec u) \\
		\rho(y) \frac{\partial^2 \vec u}{\partial t^2} - \div \boldsymbol{\sigma}(\vec u) &= \vec f  \\
		\boldsymbol{\sigma}(\vec u) &= \lambda(y) (\div \vec u) I_3 + 2\mu(y) \boldsymbol{\varepsilon}(\vec u) - (3\lambda(y) + 2\mu(y)) \alpha(y) T I_3 
	\end{aligned}
	\right.
\end{equation}

where $\rho(y)$ is the density, $c_p(y)$ is the specific heat capacity at constant pressure, $T$ is the temperature, $\kappa(y)$ is the thermal conductivity, $Q$ is the heat source, $\vec u$ is the displacement field, $\vec f$ is the body force, $\boldsymbol{\sigma}$ is the stress tensor, $\lambda(y)$ and $\mu(y)$ are the Lamé coefficients, $\boldsymbol{\varepsilon}$ is the strain tensor, $\alpha(y)$ is the thermal expansion coefficient and $T_0$ is the reference temperature.

For this study, we will consider the stationary case, where the time derivatives are zero ($\frac{\partial T}{\partial t} = 0$ and $\frac{\partial^2 \vec u}{\partial t^2} = 0$). This simplifies the system to:
\begin{equation*}
	\left\{
	\begin{aligned}
		- \div (\kappa(y) \grad T) &= Q \\
		- \div \boldsymbol{\sigma}(\vec u) &= \vec f \\
		\boldsymbol{\sigma}(\vec u) &= \lambda(y) (\div \vec u) I_3 + 2\mu(y) \boldsymbol{\varepsilon}(\vec u) - (3\lambda(y) + 2\mu(y)) \alpha(y) T I_3 
	\end{aligned}
	\right.
\end{equation*}
These stationary equations represent the equilibrium state of the thermoelastic material under constant thermal and mechanical loads.

To complete the problem formulation, we must specify the boundary conditions. Based on the problem description, we have the following conditions on the boundary $\partial\Omega$:

For the thermal problem:
\begin{align}
    T &= T_0 \quad &&\text{on } \partial\Omega_{\text{temp}} \label{eq:bc_temp_dirichlet_thermoelastic} \\
    - \kappa \grad T \cdot \vec n &= q_S \quad &&\text{on } \partial\Omega_{\text{heat}} \label{eq:bc_temp_neumann_thermoelastic}
\end{align}
where $T_0$ is the reference temperature and $q_S$ is the applied heat flux on the boundary.

For the mechanical problem:
\begin{align}
    \vec u &= \vec u_D \quad &&\text{on } \partial\Omega_D \label{eq:bc_disp_dirichlet_thermoelastic} \\
    \boldsymbol{\sigma}(\vec u) \cdot \vec n &= \vec t_N \quad &&\text{on } \partial\Omega_N \label{eq:bc_disp_neumann_thermoelastic}
\end{align}
where $\vec u_D$ is the prescribed displacement and $\vec t_N$ is the prescribed traction. 
To prove the well-posedness of the problem, we will limit the analysis to $T_0 = 0$ and $\vec u_D = \vec 0$, otherwise the problem can be rewritten with homogeneous boundary conditions with a change of variables.
 
For a given $Q \in L^2(\Omega)$, $q_S \in L^2(\partial \Omega_{\text{heat}})$, $\vec f \in \boldsymbol L^2(\Omega)$, $\vec t_N \in \boldsymbol L^2(\partial \Omega_N)$, the entire strong problem is:
\begin{tcolorbox}
Find $(T,\vec u) \in H^1(\Omega) \times \boldsymbol H^1_{\vec 0,D}(\Omega)$ such that:
\begin{align}
	- \div (\kappa(y) \grad T) &= Q && \text{on } \Omega \\
	- \div \boldsymbol{\sigma}(\vec u) &= \vec f && \text{on } \Omega \\
	\boldsymbol{\sigma}(\vec u) &= \lambda(y) (\div \vec u) I_3 + 2\mu(y) \boldsymbol{\varepsilon}(\vec u) - (3\lambda(y) + 2\mu(y)) \alpha(y) T I_3 && \text{on } \Omega \\
	\vec u &= \vec 0 && \text{on } \partial\Omega_D \\
	\boldsymbol{\sigma}(\vec u) \cdot \vec n &= \vec t_N && \text{on } \partial\Omega_N \\
	T &= 0 && \text{on } \partial\Omega_{\text{temp}} \\
	-\kappa(y) \grad T \cdot \vec n &= q_S && \text{on } \partial\Omega_{\text{heat}}
\end{align}
\end{tcolorbox}

This problem can be seen as a coupled thermo-mechanical problem, where the temperature field $T$ affects the mechanical behavior of the material through the thermal expansion term in the stress tensor $\boldsymbol{\sigma}(\vec u)$, but the mechanical displacement $\vec u$ does not directly affect the temperature field $T$. So, to solve this problem, one can first solve the thermal problem to obtain the temperature field $T$, and then use this temperature field to solve the mechanical problem for the displacement $\vec u$ with a modified source term that includes the thermal expansion effects. As $\displaystyle \forall y \in [0,22], \frac{\lambda(y)}{\mu(y)} \leq 50$, according to results from Subsection \ref{sec:mixed_BC}, we can use the pure displacement formulation to solve the mechanical problem.

\subsection{Weak Formulation}

The corresponding weak formulation is obtained by multiplying the PDEs by suitable test functions, integrating over the domain $\Omega$, and applying integration by parts.

Let the solution space for temperature be $H^1_{0,\text{temp}} = \{ \phi \in H^1(\Omega) \mid \phi = 0 \text{ on } \partial\Omega_{\text{temp}} \}$ and for displacement be $\boldsymbol{H}^1_{\vec 0,D}(\Omega)$. The weak problem is:
\begin{tcolorbox}
Find $(T, \vec u) \in H^1_{0,\text{temp}} \times \boldsymbol{H}^1_{\vec 0,D}(\Omega)$ such that for all test functions $(\theta, \vec v) \in H^1_{0,\text{temp}} \times \boldsymbol{H}^1_{\vec 0,D}(\Omega)$:
\begin{equation}
	\left\{
	\begin{aligned}
		b(T, \theta) &= l_T(\theta) \\
		a(\vec u, \vec v) &= l_U(\vec v)
	\end{aligned}
	\right.
\end{equation}
\label{eq:weak-thermoelasticity}
\end{tcolorbox}
With 
\begin{equation*}
	\begin{aligned}
	& b : \left\{
		\begin{aligned}
			& H^1_{0,\text{temp}} \times H^1_{0,\text{temp}} \longrightarrow \mtr \\
			& (T, \theta) \mapsto \int_{\Omega} \kappa(y) \grad T \cdot \grad \theta \, d\Omega 
		\end{aligned}
	\right. \\
	& l_T : \left\{
		\begin{aligned}
			& H^1_{0,\text{temp}} \longrightarrow \mtr \\
			& \int_{\Omega} Q \theta \, d\Omega - \int_{\partial\Omega_{\text{heat}}} q_S \theta \, dS
		\end{aligned}
	\right. \\
	& a : \left\{
		\begin{aligned}
			& \boldsymbol{H}^1_{\vec 0,D}(\Omega) \times \boldsymbol{H}^1_{\vec 0,D}(\Omega) \longrightarrow \mtr \\
			& (\vec u, \vec v) \mapsto \int_{\Omega} \Big( \lambda(y) (\div \vec u) (\div \vec v) + 2\mu(y) \boldsymbol{\varepsilon}(\vec u) : \boldsymbol{\varepsilon}(\vec v) \Big) \, d\Omega 
		\end{aligned}
	\right. \\
	& l_U : \left\{
		\begin{aligned}
			& \boldsymbol{H}^1_{\vec 0,D}(\Omega) \longrightarrow \mtr \\ 
			& \int_{\Omega} \vec f \cdot \vec v \, d\Omega + \int_{\partial\Omega_N} \vec t_N \cdot \vec v \, dS + \int_{\Omega} (3\lambda(y) + 2\mu(y)) \alpha(y) T (\div \vec v) \, d\Omega
		\end{aligned}
	\right.\\
	& \exists \kappa_-, \kappa_+, \lambda_-, \lambda_+, \mu_-, \mu_+, \alpha_-, \alpha_+ > 0 \text{ such that } \forall y \in [0,22], \\
	& 0 < \kappa_- \leq \kappa(y) \leq \kappa_+ < + \infty \qquad 0 < \lambda_- \leq \lambda(y) \leq \lambda_+ < + \infty \\
	& 0 < \mu_- \leq \mu(y) \leq \mu_+ < + \infty \qquad 0 < \alpha_- \leq \alpha(y) \leq \alpha_+ < + \infty.
	\end{aligned}
\end{equation*}

\noindent \textbf{Now, we prove the well-posedness of the thermal problem.} 
\begin{itemize}
    \item $(H^1_{0,\text{heat}}(\Omega),\|.\|_{H^1(\Omega)})$ is a Hilbert space.
    \item $a(\cdot,\cdot)$ is a bilinear form on $H^1_{0,\text{heat}}(\Omega)\times H^1_{0,\text{heat}}(\Omega)$ and $\ell_T$ is a linear form on $H^1_{0,\text{heat}}(\Omega)$.
    \item $\ell_T(\cdot)$ is continuous: by Cauchy--Schwarz and the trace inequality,
    \begin{equation*}
    \begin{aligned}
    \forall \theta\in H^1_{0,\text{heat}}(\Omega),\qquad
    |\ell_T(\theta)|
    &\le \|Q\|_{L^2(\Omega)}\|\theta\|_{L^2(\Omega)} + \|q_S\|_{L^2(\partial\Omega_{\mathrm{heat}})}\|\theta\|_{L^2(\partial\Omega_{\mathrm{heat}})}\\
    &\le \Big(\|Q\|_{L^2(\Omega)} + \|\gamma_0\|\,\|q_S\|_{L^2(\partial\Omega_{\mathrm{heat}})}\Big)\|\theta\|_{H^1(\Omega)},
    \end{aligned}
    \end{equation*}
    where $\gamma_0: H^1(\Omega)\to L^2(\partial\Omega_{\mathrm{heat}})$ is the trace operator.
    \item $a(\cdot,\cdot)$ is continuous:
    \begin{equation*}
    \begin{aligned}
    \forall T,\theta\in H^1_{0,\text{heat}}(\Omega),\qquad
    |a(T,\theta)|
    &\le \kappa_{+}\|\nabla T\|_{L^2(\Omega)}\|\nabla\theta\|_{L^2(\Omega)}\\
    &\le \kappa_{+}\|T\|_{H^1(\Omega)}\|\theta\|_{H^1(\Omega)}.
    \end{aligned}
    \end{equation*}
    \item $a(\cdot,\cdot)$ is coercive: $\forall T\in H^1_{0,\text{heat}}(\Omega)$,
    \begin{equation*}
    a(T,T)=\int_\Omega \kappa(y)|\nabla T|^2\,\mathrm{d}x \ge \kappa_{\min}\|\nabla T\|_{L^2(\Omega)}^2 \ge \kappa_{-} \left( \frac{1}{2}  \|\nabla T\|_{L^2(\Omega)}^2 + \frac{1}{2}\|\nabla T\|_{L^2(\Omega)}^2\right).
    \end{equation*}
    Since functions in $H^1_{0,\text{heat}}(\Omega)$ vanish on $\partial\Omega_{\mathrm{temp}}$ and a Poincaré inequality holds on $H^1_{0,\text{heat}}(\Omega)$, there exists $C_P>0$ such that $\|T\|_{L^2(\Omega)}\le C_P\|\nabla T\|_{L^2(\Omega)}$. 
    \begin{equation*}
		\begin{aligned}
		& a(T,T) \ge \kappa_{-} \left( \frac{1}{2}  \|\nabla T\|_{L^2(\Omega)}^2 + \frac{1}{2C_P^2}\|T\|_{L^2(\Omega)}^2\right) \ge \alpha\|T\|_{H^1(\Omega)}^2 \\
		& \text{With } \alpha = \min\left(\frac{\kappa_{-}}{2}, \frac{\kappa_{-}}{2C_P^2}\right) > 0.
		\end{aligned}
    \end{equation*}
\end{itemize}

Therefore $a(\cdot,\cdot)$ is continuous and coercive on the Hilbert space $H^1_{0,\text{heat}}(\Omega)$, and $\ell_T\in H^1_{0,\text{heat}}(\Omega)'$. By the Lax--Milgram theorem there exists a unique $T\in H^1_{0,\text{heat}}(\Omega)$ such that
\[
a(T,\theta)=\ell_T(\theta)\qquad\forall\theta\in H^1_{0,\text{heat}}(\Omega).
\]

The well-posedness of the displacement problem is simalar to Section \ref{sec:Pure displacement weak-formulation}. Here, we can add that $\ell_U$ is continuous, $a$ is continuous and coercive with new constants:
\begin{itemize}
	\item $\forall v \in \boldsymbol{H}^1_{\vec 0,D}(\Omega),$
	\begin{equation*}
	\begin{aligned}
	|\ell_U(v)| & \leq \left( \norm{\vec f}_{\boldsymbol L^2(\Omega)} + \norm{\gamma_0} \norm{\vec t_N}_{\boldsymbol L^2(\partial \Omega_N)}\right)\norm{\vec v}_{\boldsymbol H^1(\Omega)} + (3 \lambda_+ + 2 \mu_+) \alpha_+ \|T\|_{L^2(\Omega)} \|\div \vec v\|_{L^2(\Omega)} \\
	& \leq \left( \norm{\vec f}_{\boldsymbol L^2(\Omega)} + \norm{\gamma_0} \norm{\vec t_N}_{\boldsymbol L^2(\partial \Omega_N)}\right)\norm{\vec v}_{\boldsymbol H^1(\Omega)} + 2(3 \lambda_+ + 2 \mu_+) \alpha_+ \|T\|_{L^2(\Omega)} \|\vec v\|_{\boldsymbol H^1(\Omega)}. \\
	|\ell_U(v)| & \leq \left( \norm{\vec f}_{\boldsymbol L^2(\Omega)} + \norm{\gamma_0} \norm{\vec t_N}_{\boldsymbol L^2(\partial \Omega_N)} + 2(3 \lambda_+ + 2 \mu_+) \alpha_+ \|T\|_{L^2(\Omega)} \right) \|\vec v\|_{\boldsymbol H^1(\Omega)}.
	\end{aligned}
	\end{equation*}
	\item $\forall \vec u, \vec v \in \boldsymbol{H}^1_{\vec 0,D}(\Omega),$
	\begin{equation*}
		\left| a(\vec u, \vec v) \right| \leq (3\lambda_+ + 2\mu_+)\norm{\vec u}_{\boldsymbol H^1(\Omega)}\norm{\vec v}_{\boldsymbol H^1(\Omega)}
	\end{equation*}
	\item $\forall \vec u \in \boldsymbol{H}^1_{\vec 0,D}(\Omega),$
	\begin{equation*}
		\left| a(\vec u, \vec u) \right| \geq 2 \mu_- C \norm{\vec u}_{\boldsymbol{H}^1(\Omega)}^2
	\end{equation*}
	With $C > 0$ a constant from Korn's inequality.
\end{itemize}

\subsection{Numerical simulation}
Denotes by $V^h \subset H^1_{0,\text{heat}}(\Omega)$ a finite-dimensional subspace of test functions for temperature, $\boldsymbol{V}^h \subset \boldsymbol{H}^1_{\vec 0,D}(\Omega)$ a finite-dimensional subspace of test functions for displacement. $d$ is the degree of the B-spline spaces used to approximate the solution. Let's suppose that $(T,\vec u) \in H^{s+1}(\Omega) \times \boldsymbol{H}^{s+1}(\Omega)$ for some $s \in [1, d+1]$ is solving \eqref{eq:weak-thermoelasticity} and let's denote the corresponding approximations by $(T^h,\vec u^h) \in V^h \times \boldsymbol{V}^h$. The best approximation estimates for the B-spline spaces $V^h$ and $\boldsymbol{V}^h$ is: 
\begin{equation}
	\begin{aligned}
		& \|T - T^h\|_{H^1(\Omega)} \leq C_1 h^q \|T\|_{H^{q+1}(\Omega)}, \\
		& \|T - T^h\|_{L^2(\Omega)} \leq C_2 h^{q+1} \|T\|_{H^{q+1}(\Omega)}, \\
		& \|\vec u - \vec u^h\|_{\boldsymbol H^1(\Omega)} \leq C_3 h^q \|\vec u\|_{\boldsymbol H^{q+1}(\Omega)}, \\
		& \|\vec u - \vec u^h\|_{\boldsymbol L^2(\Omega)} \leq C_4 h^{q+1} \|\vec u\|_{\boldsymbol H^{q+1}(\Omega)}.
	\end{aligned}
	\label{eq:approximation-estimates}
\end{equation}
where $C_1, C_2, C_3, C_4 > 0$ are constants independent of $h$, and $q = \min(d, s)$ \cite{da_veiga_mathematical_2014}.
To evaluate the accuracy of the numerical solution, we can use these estimates to bound the error between the exact solution and the numerical approximation and compare it with the numerical results obtained from the simulation. Again, we use the manufactured solutions approach to evaluate the performance of the numerical method.

For that purpose, we can consider $T_e$ and $\vec u_e$ defined as: 
$$
T_e = \cos\left(\frac{\pi x}{22}\right) \cos\left(\frac{\pi y}{22}\right) \qquad \vec u_e = \begin{pmatrix}
	0 \\
	\cos\left(\frac{\pi x}{22}\right) \cos\left(\frac{\pi y}{22}\right) 
\end{pmatrix}
$$
and the corresponding source terms:
\begin{equation*}
	\begin{aligned}
	& Q = - \div(\kappa(y) \grad T_e), \qquad q_S = -\kappa(y) \grad T_e \cdot \vec n, \\
	& \vec f = -\div \boldsymbol{\sigma}(\vec u_e), \qquad \vec t_N = \boldsymbol{\sigma}(\vec u_e) \cdot \vec n.
\end{aligned}
\end{equation*}

The first result I obtained was the following plot:
\begin{figure}[!h]
	\centering
	\includegraphics[width=0.6\textwidth]{figures_thermoelas/Temp_X_max=22_Y_max=22_lim=17.0_d=2.png}
	\caption{Error plot of the temperature field ($\|T - T^h\|_{L^2(\Omega)}$ and $\|T - T^h\|_{H^1(\Omega)}$) with $d=2$ B-splines and $\text{lim} = 17.0$}
	\label{fig:error_temp_d2}
\end{figure}

In this figure, we can observe that the error in the temperature field is decreasing with the number of cells $n_{\text{cells}}$ until $n_{\text{cells}} = 32$. After that, the error is increasing and decreases again. This behavior is likely due to the interaction between the mesh refinement, the B-spline approximation properties and the properties of the medium. When the degree of the B-splines is set to 3, we can see notice in the figure \ref{fig:error_temp_d3}, that the hollow moves to a lower number of cells.

\begin{figure}[!h]
	\centering
	\includegraphics[width=0.6\textwidth]{figures_thermoelas/Temp_X_max=22_Y_max=22_lim=17_d=3.png}
	\caption{Error plot of the temperature field ($\|T - T^h\|_{L^2(\Omega)}$ and $\|T - T^h\|_{H^1(\Omega)}$) with $d=3$ B-splines and $\text{lim} = 17.0$}
	\label{fig:error_temp_d3}
\end{figure}

The same phenomenon is observed for the displacement field.
One reason that can explain this phenomenon is the placement of the B-spline knots and the corresponding basis functions. As the mesh is refined, the B-spline basis functions become more localized, which can lead to better approximation properties in certain regions of the domain. In the medium that we are considering, the line $y = \text{lim} = 17.0$ plays a crucial role in the behavior of the solution because it corresponds to a significant change in the material properties. Moreover, \texttt{PSYDAC} cannot handle coefficients defined piecewise so it is necessary to an approximation of Heaviside functions to defined coeffients. For example, $\lambda(y)$ is defined by:
$$ \lambda(y) = \big( (\lambda_W - \lambda_{Cu}) \frac{x}{\text{lim}} + \lambda_{Cu} \big) \text{H}(\text{lim}-x) + \lambda_{W} \text{H}(x-\text{lim}) \qquad H \text{ is the Heaviside function}.$$
One hypothesis that can be made to verify the role of the line $y = \text{lim}$ is to move its value up and down and observe the effect on the solution. Especially, it's possible to move the value of $\text{lim}$ to $16.5$. This value of corresponds to line with breakpoints when the domain $\Omega$ is discretized into different cells. 
\begin{figure}[!h]
	\begin{subfigure}[b]{0.49\textwidth}
		\centering
		\includegraphics[width=\textwidth]{figures_thermoelas/Temp_X_max=22_Y_max=22_lim=16.5_d=2.png}
		\caption{Error plot of the temperature field}
		\label{fig:error_temp_d2_lim_16_5}
	\end{subfigure}
	\begin{subfigure}[b]{0.49\textwidth}
		\centering
		\includegraphics[width=\textwidth]{figures_thermoelas/Disp_X_max=22_Y_max=22_lim=16.5_d=2_disp.png}
		\caption{Error plot of the displacement field}
		\label{fig:error_disp_d2_lim_16_5}
	\end{subfigure}
	\caption{Error plots with $\text{lim} = 16.5$ and degree 2 B-splines}
\end{figure}

In this case, we can observe the optimal convergence behavior of the error as predicted by the theory. So the position of the line $y = \text{lim}$ is crucial for the accuracy of the solution and the simulation.
To make the simulation possible in case described in subsection \ref{subsec:problem_setup_parameters}, we need to get a simulation setup that solve the problem when $\text{lim} = 17$.
For the moment, the function which discretised the domain and create the finite dimensional spline space takes only the number of cells per direction as argument. I have implemeted a new feature to the library which allows the user to provide a custom discretisation grid. This feature required to add new functions to the \texttt{Geometry} class and modify the previous \texttt{discretize} function in \texttt{PSYDAC}. To ensure the correct working of this new feature, I have also implemented a set of unit tests.
\begin{figure}[!h]
	\centering
	\includegraphics[width=1\textwidth]{custom_grid.png}
	\caption{Example of a custom discretisation grid}
	\label{fig:custom_grid_example}
\end{figure}
This new feature allows the user to define a custom grid for the discretisation of the domain. A entire description of this feature is available on GitHub : \href{https://github.com/pyccel/psydac/pull/518}{https://github.com/pyccel/psydac/pull/518}



\newpage
\printbibliography[heading=bibnumbered]

\newpage
\chapter{Appendices}
\section{Other plots of solutions for the problem with Dirichlet Homogeneous Boundary Conditions}



\end{document}